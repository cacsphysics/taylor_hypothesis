\documentclass[12pt,reqno]{amsart}
\usepackage[utf8]{inputenc}
\usepackage{amsmath}
\usepackage{amssymb}
\usepackage{xcolor}
\usepackage{physics}


\usepackage{lineno}


\usepackage{lineno} % adding the line numbers package for easier editing.
\usepackage{soul} % for editing purposes
\renewcommand{\linenumberfont}{\normalfont\small\color{orange}}
\setstcolor{red} % added for editing
%\usepackage[mathlines]{lineno}% Enable numbering of text and display math
\linenumbers\relax% Commence numbering lines


\definecolor{purple}{rgb}{0.6,0.4,0.8}
\newcommand{\ccedit}[1]{\textbf{\color{purple} {#1}}}
% Helpful commands to hasten the notes
%\newcommand{\epar}[1]{\partial_{#1}}
\newcommand*{\sprime}{^\ensuremath{\prime}}
\newcommand*{\dprime}{^{\ensuremath{\prime\prime}}}
\newcommand{\argument}{(\vb{x}, t)}
\newcommand{\argOne}{(x, t)}
\newcommand{\mc}[1]{\mathcal{#1}}

\newtheorem*{theorem*}{Theorem}
%% this allows for theorems which are not automatically numbered

\newtheorem{definition}{Definition}
\newtheorem{theorem}{Theorem}
\newtheorem{lemma}{Lemma}
\newtheorem{example}{Example}




\title{Studying Features of the Wave Equation}
\author{Carlos A. Cartagena-Sanchez}
\date{Updated on \today}


\begin{document}

\maketitle

\section{Introduction}
The wave equation:
\begin{equation}\label{eq:wave}
    \psi \argument = c^2\grad{\psi \argument}
\end{equation}

The general solution of the 1D wave function is the some of two counter propagating waves\[F\argOne = f_-\argOne + f_+\argOne\].
We know one solution to the wave equation is the sinusoidal functions, in particular the sine function, $\sin{(\omega t - kx)}$. The phase of the sinusoidal function is the standard linear phase function; do other phase function exists that satisfy the wave function?

Suppose we have a general phase function $A\argOne$. Injecting the now extended sinusoidal function,$f\argOne$, into the wave equation, Eq:~\ref{eq:wave}.
\textbf{Temporal Derivatives}
\begin{align*}
    \dot{f}\argOne &= \dot A\argOne \cos[ A\argOne]\\
    \ddot{f}\argOne &= \ddot{A} \argOne \cos[ A\argOne] - [\dot A\argOne]^2\sin[A\argOne]\\
\end{align*}
\textbf{Spatial Derivatives}
\begin{align*}
    f\sprime \argOne &= A\sprime \argOne \cos[ A\argOne]\\
    f{\dprime}\argOne &= A{\dprime}\argOne \cos[ A\argOne] - [A{\sprime}\argOne]^2\sin[A\argOne]\\
\end{align*}
If this extended sinusoidal function satisfies the wave equation then the following must be true.
\begin{equation}
    \frac{[\ddot A\argOne - c^2A\dprime\argOne]}{[(\dot A\argOne)^2 - c^2(A\sprime\argOne)^2]} = \tan(A\argOne)
\end{equation}
Rearranging the equation results, in what I define as, the phase wave equation.
\begin{equation}\label{eq:phase_wave}
    \ddot A\argOne - c^2A\dprime\argOne = [(\dot A\argOne)^2 - c^2(A\sprime\argOne)^2]\tan[A\argOne]
\end{equation}
We can see that if the $A\argOne$ is a wave then the phase wave equation holds. 
\begin{theorem*}[1.0]
Suppose $f\argOne$ and $g\argOne$ satisfy the wave equation, Eq:~\ref{eq:wave}. Then the composition of the functions satisfies the wave equation.
\end{theorem*}
The composition of wave functions satisfy the wave equation, meaning the composition are waves.

The question is there a class of functions that satisfy the phase wave equation, Eq.~\ref{eq:phase_wave}? The one method that comes to mind is the brute force method, polynomial expansion.
\end{document}
    